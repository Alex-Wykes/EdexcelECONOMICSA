\chapter{Theme 1: Introduction to markets and market failure}

\section{1.1 Nature of Economics}
\begin{description}
\item[MACROECONOMICS] examines the interactions between economic variables at the level of the aggregate economy.
\item[MICROECONOMICS] deals with individual decisions taken by households or firms, or in particular markets.
\end{description}

\subsection{Economics as a social science}

\begin{itemize}
\item If economists are to cope with the complexity of the real world, it is essential to simplify reality using a \textbf{model}.
\item A model almost always begins with assumptions that help economists to simplify their questions. These assumptions can be gradually relaxed, so that the effect of each one can be observed.
\item The assumptions that all other influences stay the same is commonly used by economists, this is called, \textbf{ceteris paribus}.
\item While experimental economics is expanding as a way of understanding individual behaviour, there are still many areas of economics where \textbf{it is not possible to rely on scientific experiments to advance knowledge}.
\end{itemize}
\begin{description}
\item[MODEL] a simplified representation of reality used to provide insight into economic decisions and events.
\item[CETERIS PARIBUS] a Latin phrase meaning ‘other things being equal’; it is used in economics when we focus on changes in one variable while holding other influences constant.
\end{description}

\subsection{Positive and normative economic statements \\
\textsuperscript{Syllabus Reference: 1.1.2}}
\begin{itemize}
\item A positive statement is about facts and in principle is testable; a normative statement is about what ought to be – involves a \textbf{value judgment}.
\item A model almost always begins with assumptions that help economists to simplify their questions. These assumptions can be gradually relaxed, so that the effect of each one can be observed.
\item Positive analysis is often called upon to inform normative judgments.
\item Normative value judgments do influence economic decision making and policy because different people may have different views about what is desirable for society.
\end{itemize}
\begin{description}
\item[POSITIVE STATEMENT] a statement about what is (i.e. about facts).
\item[NORMATIVE STATEMENT] a statement that involved a value judgement about what ought to be.
\end{description}

\subsection{Positive and normative economic statements \\
\textsuperscript{Syllabus Reference: 1.1.3}}
